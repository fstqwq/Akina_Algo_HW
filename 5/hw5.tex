\documentclass[11pt,a4paper,oneside]{article}
\usepackage[dvipsnames, svgnames, x11names]{xcolor}
\usepackage{euler,amssymb,amsthm,amsmath,amsfonts,graphicx,epigraph,indentfirst,enumerate,comment,listings,fontspec,color,subcaption,listings}
\usepackage{xeCJK}
\usepackage{hw}
\usepackage{pythonhighlight}
\usepackage{tikz}
\usepackage{algorithm}
\usepackage{algpseudocode}
\usepackage{float}

\newtheorem{theorem}{Theorem}
\newcommand{\nth}[1]{#1\textsuperscript{th}}
\newcommand{\E}{\mathop{\mathbb{E\/}}}
\newcommand{\R}{\mathbb{R}}
\newcommand{\norm}[1]{\|#1\|}


\renewcommand{\hwtitle} {CS217 Homework 4, First Submission}	
\renewcommand{\hwauthor}{Akina}
\renewcommand{\hwdate}{April 24, 2020}

\begin{document}
\title{\hwtitle}
\author{\hwauthor}
\date{\hwdate}
\maketitle



\setcounter{section}{4}
\section{Flows with Vertex Capacities}


\begin{problem}{1}
	\statement
    Let $G = (V,c)$ be a flow network. Prove that flow is ``transitive'' in the following sense: if $r,s,t$ are vertices, 
    and there is an $r$--$s$-flow of value $k$ and an $s$--$t$-flow of value $k$, then there is an $r$--$t$-flow of 
    value $k$.
    
    \solution
    SOLUTION
\end{problem}

\subsection{Vertex Disjoint Paths}

Let $G$ be a directed graph. Two paths $p_1, p_2$ from $s$ to $t$ are called {\em vertex disjoint}
if they don't share any vertices except $s$ and $t$. 

\begin{theorem}[Menger's Theorem]
   Let $G$ be a graph and $s \ne t$ two vertices therein. Let $k \in \mathbf{N}_0$. 
   Then exactly one of the following is true:
   \begin{enumerate}
   \item There are $k$ vertex disjoint paths $p_1,\dots,p_k$ from $s$ to $t$; that is, no two $p_i$, $p_j$ share
   any vertex besides $s$ and $t$.
   \item There are vertices $v_1,\dots,v_{k-1} \in V \setminus \{s,t\}$ such that
   $G - \{v_1,\dots, v_{k-1}\}$ contains no $s$--$t$-path.
   \end{enumerate}
\end{theorem}

\begin{problem}{3}
	\statement
   Prove Menger's Theorem. You have to prove two things: first, not both cases above can occur (this is rather easy);
   second, one of them must occur (this requires a tool from the lecture).
   
    \solution
    SOLUTION
\end{problem}



Let $V = \{0,1\}^n$. The $n$-dimensional Hamming cube $H_n$ is the graph $(V,E)$ where
$\{u,v\} \in E$ if $u,v$ differ in exactly one coordinate.
Define the $\nth{i}$ level of $H_n$ as 
\begin{align*}
  L_i := \{u \in V \ | \ \norm{u}_1 = i \} \ ,
\end{align*}
i.e., those vertices $u$ having exactly $i$ coordinates which are $1$.
\begin{center}
  \includegraphics[width=0.8\textwidth]{figures/hamming-3-dim.pdf}\\
  {\small The $3$-dimensional Hamming cube and the four 
    sets $L_0$, $L_1$, $L_2$, $L_3$.}
\end{center}


\begin{problem}{4}
  \statement
  Consider the induced bipartite subgraph $H_n[ L_i \cup L_{i+1}]$. This is 
  the graph on vertex set $L_i \cup L_{i+1}$ where two edges are connected
  by an edge if and only if they are connected in $H_n$.
  Show that for $i < n/2$ the graph $H_n[ L_i \cup L_{i+1}]$
  has a matching of size $|L_i| = {n \choose i}$.
  
  \solution
  SOLUTION
\end{problem}


\begin{problem}{5}
	\statement
  Let $H_n$ be the $n$-dimensional Hamming cube. For $i < n/2$ consider
  $L_i$ and $L_{n-i}$. Note that 
  $|L_i| = {n \choose i} = { n \choose n-i}  = L_{n-i}$, so the 
  $L_i$ and $L_{n-i}$ have the same size.   Show that there are ${n \choose i}$ paths $p_1,p_2,\dots,p_{ {n \choose i}}$
  in $H_n$ such that
  (i) each $p_i$ starts in $L_i$ and ends in $L_{n-i}$;
  (ii) two different paths $p_i,p_j$ do not share any vertices.
  
    \solution
    SOLUTION
\end{problem}


\subsection{Matchings and Vertex Covers}

The following exercise was on the final exam of CS 499 (mathematical foundations of computer science) in spring 2019.

\begin{problem}{6}
	\statement
    Let $\nu(G)$ denote the size of a maximum matching of $G$. Show that a bipartite graph $G$
    has at most $2^{\nu(G)}$ minimum vertex covers.
    
    \solution
    Select a maximum matching $M$ of $G$. 
    Edges in $M$ don't share any vertex, so a minimum vertex cover should include at least one vertex in each edge of $M$. 
	And because $G$ is a bipartite graph, Konig's Theorem implies that $\nu(G)$ = $\min|C|$. 
    So a minimum vertex cover should include exactly one vertex in each edge of $M$, and doesn't include any other vertex. 
    A minimum vertex cover has at most 2 choices to cover a edge in $M$ and $M$ has $\nu(G)$ edges, 
    so $G$ has at most $2^{\nu(G)}$ minimum vertex covers. 
\end{problem}

Obviously, this is not  true for general (non-bipartite) graphs: the triangle $K_3$ has $\nu(K_3) = 1$ but it has 
three minimum vertex covers. The five-cycle $C_5$ has $\nu(C_5) = 2$ but has five minimum vertex covers.

\begin{problem}{7}
	\statement
   Is there a function $f: \mathbf{N}_0 \rightarrow \mathbf{N}_0$ such that every graph with $\nu(G) = k$ has 
   at most $f(k)$ minimum vertex covers? How small a function $f$ can you obtain?
   
    \solution   
    Select a maximum matching $M$ of $G$ with $|M|=k$. 
    If both vertex of an edge in $G$ aren't included in $M$, then this edge can be added into $M$ to make a larger matching, 
    so every edge of $G$ has at least 1 vertex in $M$.
    
	
	For a minimum vertex cover $C$,define $C\cap M = N$ and $G'=\{ e\in E | e\cup N=\emptyset\}$. 
	Because every edge of $G$ has at least 1 vertex in $M$, all edges in $G'$ has at least 1 vertex in $M/N$, which means the vertex won't appear in $C$. 
	If there is a edge in $G'$ with both vertexes in $M/N$, then this edge isn't covered by $C$. 
	For every edge in $G'$ with one vertex in $M/N$, the other vertex should be included in $C$ to cover this edge, 
	and by adding these vertexes a minimum vertex cover is made.
	Therefore, $C$ can be generated with $N$ in a definite way.

	Because $M$ has k edges and in each edge at least 1 vertex should be included to cover itself, there exists at most $3^k$ ways to select $N$ from $M$.
	Therefore the number of minimum vertex covers is $3^k$ at most.
	$f(k)=3^k$ can be obtained, and it's exactly the number of minimum vertex covers when $G$ is composed of k triangle $K_3$.
\end{problem}


\end{document}