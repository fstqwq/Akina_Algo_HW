\documentclass[11pt,a4paper,oneside]{article}
\usepackage[dvipsnames, svgnames, x11names]{xcolor}
\usepackage{euler,amsthm,amsmath,amsfonts,graphicx,epigraph,indentfirst,enumerate,comment,listings,fontspec,color,subcaption,listings}
\usepackage{xeCJK}
\usepackage{hw}
\usepackage{pythonhighlight}
\usepackage{tikz}
\usepackage{algorithm}
\usepackage{algpseudocode}
\usepackage{float}

\newcommand{\nth}[1]{#1\textsuperscript{th}}
\newcommand{\E}{\mathop{\mathbb{E\/}}}
\newcommand{\R}{\mathbb{R}}


\renewcommand{\hwtitle} {CS217 Homework 2, Final Submission}	
\renewcommand{\hwauthor}{Akina}
\renewcommand{\hwdate}{April 3, 2020}

\begin{document}
\title{\hwtitle}
\author{\hwauthor}
\date{\hwdate}
\maketitle

\section*{Bottleneck Paths}

Let $G=(V,E)$ be a directed graph with an edge capacity function $c: E \rightarrow \R^+$. For a path
$p = u_0 u_1 \dots u_t$ define its {\em capacity} to be
\begin{align}
c(p) := \min_{1 \leq i \leq t} c( \{u_{i-1}, u_i\}) \ .
\end{align}

\begin{quotation}
	\textbf{Maximum Capacity Path Problem (MCP).} Given a directed graph $G = (V,E)$, an edge capacity function
	$c: E \rightarrow \R^+$, and two vertices $s, t \in V$, compute the path $p^*$ maximizing $c(p)$. We
	denote by $p^*$ the optimal path and by $c^* := c(p^*)$ its cost. 
\end{quotation}

\begin{problem}{1}
	\statement
	Suppose the edges $e_1,\dots,e_m$ are sorted by their capasity. Show how to solve MCP in time $O(n+m)$.
	\solution
	Assume that \(c(e_1) \geq c(e_2) \geq \dots, \geq c(e_m)\). Each time we add an edge in order and check whether \(s\) and \(t\) are connected. If so, we stop. Assume that we stop at edge \(e_x\), \(c^* = c(e_x)\) and we can use dfs algorithm to find a path from \(s\) to \(t\) in the graph which contains \(e_1, e_2, \dots, e_x\).
	
	Next we anylise its complexity. In worst case, we need to add \(m\) edges and check \(m\) times. The complexity \(O(n + m)\) if we use an algorithm similar to coloring. 
	
	In the beginning, each node has no color except \(s\) is red and \(t\) is blue. Once an edge is added, the color will spread along the edge. If two kind of color are mixed, we can claim that \(s\) and \(t\) are connected.
	
	Each edge will be add once, and each node will be colored once, hence the complexity is \(O(n + m)\)
	
	Together with the complexity of dfs, in conclusion, we can solve MCP in time \(O(n + m)\)
\end{problem}

\begin{problem}{2}
	\statement
	Give an algorithm for MCP of running time $O(m \log \log m)$. \textbf{Hint:} Using the median-of-medians algorithm,
	you can determine an edge $e$ such that at most $m/2$ edges are cheaper than $e$ and at most $m/2$ edges are
	more expensive than $e$. Can you determine, in time $O(n+m)$, whether $c^* < c(e)$, $c^* = c(e)$, or $c^* > c(e)$?
	Iterate to shrink the set of possible
	values for $c^*$ to $m/4$, $m/8$, and so on.

	\solution
	
\end{problem}
\begin{problem}{3}
	\statement
	Give an algorithm for MCP that runs in time $O(m \log \log \log m)$? How about $O(m \log \log \log \log m)$? How far can you get?

\end{problem}

\begin{problem}{4}
\statement
	Suppose we modify the Ford-Fulkerson method so that, in every round, it finds a path of {\em maximum capacity}, as opposed
	to be shortest-$s-t$-path method employed by Edmonds-Karp. Show that this algorithm terminates after a number of rounds that is 
	polynomial in $n$ and $m$.
\solution
\end{problem}

\end{document}
