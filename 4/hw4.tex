\documentclass[11pt,a4paper,oneside]{article}
\usepackage[dvipsnames, svgnames, x11names]{xcolor}
\usepackage{euler,amssymb,amsthm,amsmath,amsfonts,graphicx,epigraph,indentfirst,enumerate,comment,listings,fontspec,color,subcaption,listings}
\usepackage{xeCJK}
\usepackage{hw}
\usepackage{pythonhighlight}
\usepackage{tikz}
\usepackage{algorithm}
\usepackage{algpseudocode}
\usepackage{float}

\newcommand{\nth}[1]{#1\textsuperscript{th}}
\newcommand{\E}{\mathop{\mathbb{E\/}}}
\newcommand{\R}{\mathbb{R}}


\renewcommand{\hwtitle} {CS217 Homework 4, First Submission}	
\renewcommand{\hwauthor}{Akina}
\renewcommand{\hwdate}{April 24, 2020}

\begin{document}
\title{\hwtitle}
\author{\hwauthor}
\date{\hwdate}
\maketitle

\section*{Bottleneck Paths}

Let $G=(V,E)$ be a directed graph with an edge capacity function $c: E \rightarrow \R^+$. For a path
$p = u_0 u_1 \dots u_t$ define its {\em capacity} to be
\begin{align}
c(p) := \min_{1 \leq i \leq t} c( \{u_{i-1}, u_i\}) \ .
\end{align}

\begin{quotation}
	\textbf{Maximum Capacity Path Problem (MCP).} Given a directed graph $G = (V,E)$, an edge capacity function
	$c: E \rightarrow \R^+$, and two vertices $s, t \in V$, compute the path $p^*$ maximizing $c(p)$. We
	denote by $p^*$ the optimal path and by $c^* := c(p^*)$ its cost. 
\end{quotation}

\begin{problem}{1}
	\statement
	Suppose the edges $e_1,\dots,e_m$ are sorted by their capasity. Show how to solve MCP in time $O(n+m)$.
	\solution
	Assume edges sorted from higher capacity to lower. Add edge one by one and check whether \(s\) and \(t\) are connected. If so, report the last edge's capacity as the maximum capacity $C$, and it's easy to report the path with such capacity simply use searches from $s$ to $t$. See \textbf{Algorithm 1} pseudo code for detail.

	The correctness is straight forward: for any path $p$ with higher capacity,it is impossible that $p$ start from $s$ and end at $t$ since they are disconnected. For the current graph with all edges with capacity at least $C$, since every simple path holds a capacity at least $C$ and $s$ and $t$ is connected, there's at least one path with such capacity $C$ connecting from $s$ to $t$.

	\begin{algorithm}
		\caption{Find MCP using sorted edges in $O(n + m)$}
		\begin{algorithmic}[1]
			\Function{MCP}{$V, E, c, s, t$}
			\State $S \leftarrow \{ s \}$
			\State $E',Relax \leftarrow \varnothing$
			\For {$e \in E$}
				\State $AddEdge(e)$
				\If {$t \in S$}
					\State \Return $\{c(e), FindPath(V, E', s, t)\}$
					\Comment FindPath finds path $s \rightarrow t$ using trivial DFS or BFS
				\EndIf
			\EndFor
			\State \Return NO SOLUTION
			\EndFunction
			\Function{AddEdge}{$e$}
				\State $E' \leftarrow E' \cup e$
				\State $Relax \leftarrow Relax \cup e$
				\If {$e.from \in S$}
					\State $COLOR(e.from)$
				\EndIf
			\EndFunction
			\Function{COLOR}{$x$}
				\State $S \leftarrow S \cup \{ x \}$
				\For {$e = (x, y) \in Relax$}
				\State $Relax \leftarrow Relax \backslash e$
				\State $COLOR(y)$
				\EndFor
			\EndFunction
		\end{algorithmic}
	\end{algorithm}
	
	The complexity, too, is straight forward. Every edge will enter and exit $Relax$ at most once, say $O(m)$. For function $COLOR$ it will run at most $2m$ times since every edge update may cause a call of $COLOR$, and exiting $Relax$ may cause a call of $COLOR$. $FindPath$ is simply search a path using bfs or visit-once dfs, which is $O(n + m)$. Thus the algorithm solves MCP in time \(O(n + m)\).
\end{problem}

\begin{problem}{2}
	\statement
	Give an algorithm for MCP of running time $O(m \log \log m)$. \textbf{Hint:} Using the median-of-medians algorithm,
	you can determine an edge $e$ such that at most $m/2$ edges are cheaper than $e$ and at most $m/2$ edges are
	more expensive than $e$. Can you determine, in time $O(n+m)$, whether $c^* < c(e)$, $c^* = c(e)$, or $c^* > c(e)$?
	Iterate to shrink the set of possible
	values for $c^*$ to $m/4$, $m/8$, and so on.

	\solution
	
	\begin{algorithm}
		\caption{Find MCP in $O(m \log^* m)$ using median of medians}
		\begin{algorithmic}[1]

			\Function{MCP}{$V, E, c, s, t$}
				\If {$s, t$ are connected in $(V, E)$}
					\State \Return $MCP(V, E, c, s, t, \{s\}, \varnothing, 1)$
				\Else
					\State \Return NO SOLUTION
				\EndIf
			\EndFunction

			\Function{Solve}{$V, E, c, s, t, S, D, k$}
			\If {$|c(E)| = 1$}
				\State \Return $\{c(E_0), FindPath(V, E \cup D, s, t)\}$
			\EndIf
			
			\State $P \gets$ $\{0\} \cup GetPivot(c(E), k) \cup  \{+\infty\}$ 
			\State $E_i \gets \{e \in E | P_i < c(e) \leq P_{i+1} \}$

			\For {$ i= |P| - 2 \to 0$}  
				\State $S' \gets$ the connected component of $S$ in $(V, D \cup E_i)$
				\If {$t \in S'$}
					\State \Return {$Solve(V, E_i, c, s, t, S, D, k 2 ^ k)$}
				\EndIf
				\State $D \gets D \cup E_i$
				\State $S \gets S'$
			\EndFor
			\EndFunction
			\Function{GetPivot}{$S, k$}
				\If {$S = \varnothing$}
					\State \Return $\varnothing$
				\EndIf
				\State $P \leftarrow MedianOfMedians(S)$
				\If {$k = 1$ or $|S| = 1$}
					\State \Return $\{P\}$
				\Else
					\State \Return $GetPivot(\{s | s < P\}, k - 1) \cup \{P\} \cup GetPivot(\{s | s > P\}, k - 1)$
				\EndIf
			\EndFunction
		\end{algorithmic}
	\end{algorithm}


	The pseudo code \textbf{Algorithm 2} described our solution. The main idea is: on each level we divided edges in to certain blocks using median of median in order to achieve a total $O(m)$. Adding edges in blocks like \textbf{Algorithm 1} until $s, t$ are connected, and recursively solve the subproblem in the current block, while keeping determined edges in previous blocks.

	Consider the complexity of such algorithm. Each iteration, say current block size is $m / k$, it takes $O(m)$ to divide current block into $2 ^ k$ parts applying MoM recursively, and $k$ increase to $k 2 ^ k$ in the recursion. When $k \geq |E|$ the algorithm will stop in at most $1$ iteration, since edges are equal capacity in each block. Note that $k \rightarrow k 2 ^ k$ grows extremely fast, for example

	$$ 1 \rightarrow 2 \rightarrow 8 \rightarrow 2048 \rightarrow 2 ^ {2060} \rightarrow \cdots$$
	
	We have a better notation for it, that \(\lceil \log^* m \rceil\), defined as $$
		\log ^ * n =
		\begin{cases}
			0, &n \leq 1 \\
			1 + \log^*(\log n), &n > 1
		\end{cases}
	$$
	
	So the total complexity is \(O(mlog^*m)\), which is better than \(O(m\log\log m)\) and \(O(m\log\log\log m)\) and so on.
	
\end{problem}
\begin{problem}{3}
	\statement
	Give an algorithm for MCP that runs in time $O(m \log \log \log m)$? How about $O(m \log \log \log \log m)$? How far can you get?
	
	\solution
	
	See \textbf{Problem 2}.

\end{problem}

\end{document}
