\documentclass[11pt,a4paper,oneside]{article}
\usepackage[dvipsnames, svgnames, x11names]{xcolor}
\usepackage{euler,amsthm,amsmath,amsfonts,graphicx,epigraph,indentfirst,enumerate,comment,listings,fontspec,color,subcaption,listings}
\usepackage{xeCJK}
\usepackage{hw}
\usepackage{pythonhighlight}
\usepackage{tikz}
\usepackage{algorithm}
\usepackage{algpseudocode}
\usepackage{float}

\newcommand{\nth}[1]{#1\textsuperscript{th}}
\newcommand{\E}{\mathop{\mathbb{E\/}}}
\newcommand{\R}{\mathbb{R}}


\renewcommand{\hwtitle} {CS217 Homework 2, Final Submission}	
\renewcommand{\hwauthor}{Akina}
\renewcommand{\hwdate}{April 3, 2020}

\begin{document}
\title{\hwtitle}
\author{\hwauthor}
\date{\hwdate}
\maketitle

\section*{Bottleneck Paths}

Let $G=(V,E)$ be a directed graph with an edge capacity function $c: E \rightarrow \R^+$. For a path
$p = u_0 u_1 \dots u_t$ define its {\em capacity} to be
\begin{align}
c(p) := \min_{1 \leq i \leq t} c( \{u_{i-1}, u_i\}) \ .
\end{align}

\begin{quotation}
	\textbf{Maximum Capacity Path Problem (MCP).} Given a directed graph $G = (V,E)$, an edge capacity function
	$c: E \rightarrow \R^+$, and two vertices $s, t \in V$, compute the path $p^*$ maximizing $c(p)$. We
	denote by $p^*$ the optimal path and by $c^* := c(p^*)$ its cost. 
\end{quotation}

\begin{problem}{1}
	\statement
	Suppose the edges $e_1,\dots,e_m$ are sorted by their capasity. Show how to solve MCP in time $O(n+m)$.
	\solution
	Assume that \(c(e_1) \geq c(e_2) \geq \dots, \geq c(e_m)\). Each time we add an edge in order and check whether \(s\) and \(t\) are connected. If so, we stop. Assume that we stop at edge \(e_x\), \(c^* = c(e_x)\) and we can use dfs algorithm to find a path from \(s\) to \(t\) in the graph which contains \(e_1, e_2, \dots, e_x\).
	
	Next we analyse its complexity. In worst case, we need to add \(m\) edges and check \(m\) times. The complexity \(O(n + m)\) if we maintain a set \(S\) of vertices which is reachable from \(s\) and a set of edges \(P\) whose endpoints not in \(S\).
	
	In the beginning, \(S\) only contains \(s\). Once an edge is added, if one of its endpoints in \(S\) but the other not, do dfs algorithm on \(V \setminus S\) from the endpoint not in \(S\). Add all the vertices involved to \(S\).
	
	Each edge will be add to \(S\) once, and each node will be travelled once, hence the complexity is \(O(n + m)\)
	
	Together with the complexity of dfs, in conclusion, we can solve MCP in time \(O(n + m)\)
\end{problem}

\begin{problem}{2}
	\statement
	Give an algorithm for MCP of running time $O(m \log \log m)$. \textbf{Hint:} Using the median-of-medians algorithm,
	you can determine an edge $e$ such that at most $m/2$ edges are cheaper than $e$ and at most $m/2$ edges are
	more expensive than $e$. Can you determine, in time $O(n+m)$, whether $c^* < c(e)$, $c^* = c(e)$, or $c^* > c(e)$?
	Iterate to shrink the set of possible
	values for $c^*$ to $m/4$, $m/8$, and so on.

	\solution
	
	\begin{algorithm}
		\caption{find MCP in $O(\log^* n)$}
		\begin{algorithmic}[1]
			\Function{MCP}{$V, E, s, t, c, S, E', k$}
			\If $|E| = 1$
				\If {some $s$ and $t$ are connected in $(V, E \cup E')$}
					\State \Return The path from $s$ to $t$, $c(E_0)$
				\Else
					\State \Return No MCP
				\EndIf
			\EndIf
			
			\State $P \gets$ $\{0\} \cup$ median of median $k$ times on $c(E)$
			\State $E_i \gets \{e \in E | P_i \leq c(e) \leq P_{i+1} \}$

			\For {$ i= 2^k \to 0$}  
				\State $S' \gets$ the connected component of $S$ in $(V, E' \cup E_i)$
				\If {$t \in S'$}
					\State \Return {$MCP(V, E', s, t, c, S, E', k+1)$}
				\EndIf
				\State $E' \gets E' \cup E_i$
				\State $S \gets S'$
			\EndFor
			\State \Return No MCP
			\EndFunction
		\end{algorithmic}
	\end{algorithm}

	Now we analyze the alogirhtm's complexity. Each recursion the edge set \(E\) is divided into \(2^k\) parts, where \(k\) is the number of recursions. We achive it by choosing \(2^k\) pivot using median of median algorithm, which can be done in \(O(m)\). To check whether \(t \in S\), we can use the algorithm in \textbf{Problem 1}, by which we only need \(O(n+m)\) time to check it in a series of edge adding operations. Hence each recursoin, we need \(O(m)\) time to have all this done.
	
	What important is the number of recursions. Assume that the number is \(k\). \(k\) is the smallest integer greater than \(0\) and satisfying:
	
	\[2^{2^{2^{2^{\dots}}} \left\{k \text{times} \right.} \geq m\]
	
	In other word, the number of recursions is \(\log^* m\). So the total complexity is \(O(mlog^*m\), which is better than \(O(m\log\log m)\) and \(O(m\log\log\log m)\) and so on.
	
\end{problem}
\begin{problem}{3}
	\statement
	Give an algorithm for MCP that runs in time $O(m \log \log \log m)$? How about $O(m \log \log \log \log m)$? How far can you get?
	
	\solution
	
	See \textbf{Problem 2}.

\end{problem}

\end{document}
