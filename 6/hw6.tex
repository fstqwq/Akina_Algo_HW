\documentclass[11pt,a4paper,oneside]{article}
\usepackage[dvipsnames, svgnames, x11names]{xcolor}
\usepackage{euler,amssymb,amsthm,amsmath,amsfonts,graphicx,epigraph,indentfirst,enumerate,comment,listings,fontspec,color,subcaption,listings}
\usepackage{xeCJK}
\usepackage{hw}
\usepackage{pythonhighlight}
\usepackage{tikz}
\usepackage{algorithm}
\usepackage{algpseudocode}
\usepackage{float}

\newtheorem{theorem}{Theorem}
\newtheorem{definition}[theorem]{Definition}
\newcommand{\nth}[1]{#1\textsuperscript{th}}
\newcommand{\E}{\mathop{\mathbb{E\/}}}
\newcommand{\R}{\mathbb{R}}
\newcommand{\x}{\mathbf{x}}
\newcommand{\y}{\mathbf{y}}
\newcommand{\sol}{\textup{\textrm{sol}}}
\newcommand{\val}{\textnormal{val}}
\newcommand{\opt}{\textup{\textrm{opt}}}
\newcommand{\norm}[1]{\|#1\|}
\newcommand{\rank}{\textnormal{rank}}
\renewcommand{\a}{\mathbf{a}}
\renewcommand{\b}{\mathbf{b}} 


\renewcommand{\hwtitle} {CS217 Homework 5, First Submission}	
\renewcommand{\hwauthor}{Akina}
\renewcommand{\hwdate}{May 8, 2020}

\begin{document}
\title{\hwtitle}
\author{\hwauthor}
\date{\hwdate}
\maketitle


\section*{Matching LP and Vertex Cover LP}



Let $G=(V,E)$ be a 
graph and consider the Vertex Cover Linear Program $\textrm{VCLP}(G)$:
\begin{align*}
\textrm{VCLP($G$)}: \qquad
\begin{array}{lrl}
\textnormal{minimize} \quad & \multicolumn{2}{l}{\sum_{u \in V} y_u} \\
\textnormal{subject to} \quad & y_u + y_v  & \geq 1 \quad \forall\ \textnormal{ edges } \{u,v\} \in E\\
& \y & \geq \mathbf{0}  
\end{array}
\end{align*}
Every vertex cover of $G$ corresponds to a feasible solution $\y \in \sol(\textrm{VCLP}(G))$, but not 
vice versa. However, every $\y \in \sol(\textrm{VCLP}(G)) \cap \{0,1\}^V$ does.
Let $\tau(G)$ denote the size of a minimum vertex cover of $G$. In class, we showed that
$\tau(G) = \val(\textrm{VCLP}(G))$ for all {\em bipartite} graphs $G$. We achieved this by taking
an arbitrary feasible solution $\y$ and ``shaking'' it until it becomes integral, while making sure
its value does not go up.\\


Next, recall the Matching Linear Program $\textrm{MLP}(G)$:
\begin{align*}
\textrm{MLP($G$)}: \qquad
\begin{array}{lrl}
\textnormal{maximize} \quad & \multicolumn{2}{l}{\sum_{e \in E} x_e} \\
\textnormal{subject to} \quad & \sum_{e \in E: u \in e} x_e  & \leq 1 \quad \forall\ u \in V \\
& \x & \geq \mathbf{0}  
\end{array}
\end{align*}
Every matching of $G$ corresponds to a feasible solution $\x \in \sol(\textrm{MLP}(G))$, but not 
vice versa. However, every $\x \in \sol(\textrm{MLP}(G)) \cap \{0,1\}^E$ does.

\begin{problem}{1}
	\statement
   Let $\nu(G)$ denote the size of a maximum matching of $G$. 
   Obviously, $ \val(\textrm{MLP}(G)) \geq \nu(G)$ for all graphs.
   Show that
   $\nu(G) = \val(\textrm{MLP}(G))$ for all {\em bipartite} graphs $G$.
   
    \solution
    
\end{problem}

\begin{problem}{2}
  \statement
 We know that $\nu(G) = \tau(G)$ for all bipartite graphs (K\H{o}nig's Theorem) and
 $\nu(G) \leq \tau(G)$ for all graphs (since every matched edge must be covered
 by a distinct vertex). Show that $\tau(G) \leq 2 \, \nu(G)$ for all graphs $G$.   
  
  \solution
  
  Let \(M\) be a maximum matching of graph \(G\) and \(W := \{u, v | \{u, v\} \in M\}\). There is no such edge \(\{u, v\} \in E \setminus M\) satisfying that \(u\) and \(v\) both in \(V \setminus W\), otherwise \(\{u, v\}\) should be in \(M\). Therefore there are two types of edges in \(E \setminus M\):
  
  \begin{itemize}
  	\item \(\{u, v\}\) which \(u, v\) are all \(\in W\)
  	\item \(\{u, v\}\) which only one of \(u\) and \(v\) \(\in W\)
  \end{itemize}

	So, \(W\) is a vertex cover, and \(|W| = 2\nu(G)\), hence \(\tau(G) \leq |W| \leq 2\nu(G)\).
\end{problem}


\begin{problem}{3}
	\statement
     Show that $\tau(G) \leq 2\, \opt(\textrm{VCLP}(G))$ for all graphs $G$ (including non-bipartite graphs).
  
    \solution
    
\end{problem}



\textbf{Basic Solutions.} Recall our definition of basic solutions.
Let $P$ be the following linear program.
\begin{align*}
P: \qquad
\begin{array}{lrl}
\textnormal{maximize} \quad & \multicolumn{2}{l}{\mathbf{c}^T \x} \\
\textnormal{subject to} \quad & A\x & \leq \b \ 
\end{array}
\end{align*}
where we translated the constraint $\x \geq 0$ into $n$ constraints
$-x_i \leq 0$ and integrated them into $A$, so 
the $n$ last rows of $A$ form the 
negative identity matrix $-I_n$. We introduce some notation: $\a_i$ is the $\nth{i}$ row of $i$;
for $I \subseteq [m+n]$ let $A_I$ be the matrix
consisting of the rows $\a_i$ for $i \in I$.
\begin{definition}
	For $\x \in \R^n$ let $I(\x) := \{i \in [m+n] \ | \ \a_i \x = b_i\}$
	be the set of indices of the constraints that are ``tight'', i.e.,
	satisfied with equality (we include non-negativity constraints here).  We call $\x \in \R^n$ a {\em basic point}
	if $\rank(A_{I(x)}) = n$.  If $\x$ is a basic point and feasible, we
	call it a {\em basic feasible solution} or simply a {\em basic
		solution}.
\end{definition}
We can define the same concept for minimization programs. \\

We say a set $C \subseteq V$ is a minimal vertex cover of $G = (V,E)$ if 
(1) it is a vertex cover and (2) it is minimal, i.e., for every $u \in C$ the set
$C \setminus \{u\}$ is not a vertex cover anymore.

\begin{problem}{4}
	\statement
     Let $G$ be a bipartite graph.
    Show that $\y \in \R^{V}$ is a basic solution of ${\rm VCLP}(G)$ if and only if 
    (1) $y_u \in \{0,1\}$ for all $u \in V$ and (2) the set $C := \{u \in V \ | \ y_u = 1\}$ is a 
    minimal vertex cover.
    
    \solution
    
\end{problem}

\textbf{Hint.} Suppose $e_1,\dots,e_k$ form a cycle in $G$. Note that every edge corresponds
to a constraint of the VCLP, and thus this cycle corresponds to a submatrix $A_I$ with
$|I|= k$. Show that the $k$ rows of $A_I$ are linearly dependent. 

\textbf{Hint.} Suppose $C$ is a minimal vertex cover. Let $F$ be the set of ``tight'' edges, i.e., 
the edges $e \in E$ incident to exactly one $u \in C$. What does minimality of $C$ say 
about the relation between $C$ and $F$? Does this help you to show that the set
of tight constraints of VCLP has rank $n$?

\textbf{Hint.} Conversely, suppose $\y$ is a basic solution. Look at the vertices
$u$ with $y_0 = 0$ and the ``tight edges'', those $e = \{u,v\}$ for which $y_u + y_v = 1$.
\end{document}