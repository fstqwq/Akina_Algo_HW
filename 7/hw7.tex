\documentclass[11pt,a4paper,oneside]{article}
\usepackage[dvipsnames, svgnames, x11names]{xcolor}
\usepackage{euler,amssymb,amsthm,amsmath,amsfonts,graphicx,epigraph,indentfirst,enumerate,comment,listings,fontspec,color,subcaption,listings}
\usepackage{xeCJK}
\usepackage{hw}
\usepackage{pythonhighlight}
\usepackage{tikz}
\usepackage{algorithm}
\usepackage{algpseudocode}
\usepackage{float}


\newcommand{\rank}{\textnormal{rank}}
\newcommand{\y}{\mathbf{y}}
\renewcommand{\c}{\mathbf{c}}
\newcommand{\x}{\mathbf{x}}
\newcommand{\z}{\mathbf{z}}
\renewcommand{\u}{\mathbf{u}}
\newcommand{\V}{\mathbf{v}}
\renewcommand{\a}{\mathbf{a}}
\renewcommand{\b}{\mathbf{b}} 
\newcommand{\zero}{\mathbf{0}}
\newcommand{\rpn}{\mathbb{R}_{\geq 0}}
\newcommand{\sol}{\textup{\textrm{sol}}}
\newcommand{\opt}{\textup{\textrm{opt}}}

\newtheorem{theorem}{Theorem}
\newcommand{\nth}[1]{#1\textsuperscript{th}}
\newcommand{\E}{\mathop{\mathbb{E\/}}}
\newcommand{\R}{\mathbb{R}}
\newcommand{\norm}[1]{\|#1\|}


\renewcommand{\hwtitle} {CS217 Homework 7, Final Submission}	
\renewcommand{\hwauthor}{Akina}
\renewcommand{\hwdate}{\today}

\begin{document}
\title{\hwtitle}
\author{\hwauthor}
\date{\hwdate}
\maketitle


\section{Farkas Lemma and LP Duality}

\subsection{Different Versions of Farkas Lemma}

\begin{problem}{1}
	\statement
 Show that the three versions of Farkas Lemma presented in class are all equivalent (I actually did not present
 the third version in class):
 \begin{align}
   ( \neg \exists \x : \, A \x \leq \b ) \, & \Longleftrightarrow 
    ( \exists \y \geq \zero : \, \y^T A = \zero, \,     \y^T \b < 0 ) \ . \\
      ( \neg \exists \x \geq \zero : \, A \x \leq \b ) \, & \Longleftrightarrow 
    ( \exists \y \geq \zero : \, \y^T A \geq \zero, \,  \y^T \b < 0 ) \ . \\
   ( \neg \exists \x \geq \zero : \, A \x = \b ) \, & \Longleftrightarrow 
    ( \exists \y \begingroup \color{white} \geq \zero \endgroup : \, \y^T A \geq \zero, \,  \y^T \b < 0 ) \ .
 \end{align}
  Note that the direction ``$\Longleftarrow$'' is easy in each case. 
  We will show the ``$\Longrightarrow$'' of (1) in class using a technique called {\em Fourier-Motzkin Elimination}. 
  This exercise is actually not that hard. The hardest part is keeping track of what you 
  want to prove and what you can assume.
    
    \solution
	Proof from (1) to (3) is:
	
	Define $A'=\begin{bmatrix}A\\-A\\-1\end{bmatrix}$,
	$b'=\begin{bmatrix}b\\-b\\0\end{bmatrix}$,
	$y'=\begin{bmatrix}y_1\\y_2\\z\end{bmatrix}$,then:
	\begin{align*}
	(\neg\exists x\geq 0:Ax=b)
	& \Longleftrightarrow (\neg\exists x\geq 0:Ax\leq b,-Ax\leq -b,-x\leq 0)\\
	& \Longleftrightarrow (\neg\exists x\geq 0:A'x\leq b)\\
	& \Longleftrightarrow (\exists y'\geq 0:y'^TA'=0,y'^Tb'<0)\\
	& \Longleftrightarrow (\exists y_1,y_2,z\geq 0:(y_1^T-y_2^T)A'-z=0,(y_1^T-t_2^T)b<0)\\
	& \Longleftrightarrow (\exists z\geq 0,y:y'^TA'=z,y^Tb<0)\\
	& \Longleftrightarrow (\exists y\geq 0,y:y'^TA'\geq 0,y^Tb<0)	
	\end{align*}
	
	Proof from (3) to (2) is:
	
	Define $A'=\begin{bmatrix}A & 1\end{bmatrix}$,
	$x'=\begin{bmatrix}x\\z\end{bmatrix}$,then:
	\begin{align*}
	(\neg\exists x\geq 0:Ax\leq b)
	& \Longleftrightarrow (\neg\exists x,z\geq 0:Ax+z=b)\\
	& \Longleftrightarrow (\neg\exists x\geq 0:A'x'=b)\\
	& \Longleftrightarrow (\exists y:y^TA'\geq 0,y^Tb<0)\\
	& \Longleftrightarrow (\exists y\geq 0:y^TA\geq q0,y^Tb<0)
	\end{align*}
	
	Proof from (2) to (1) is:
	
	Define $A'=\begin{bmatrix}A & -A\end{bmatrix}$,
	$x'=\begin{bmatrix}x_1\\x_2\end{bmatrix}$,then:
	\begin{align*}
	(\neg\exists x:Ax\leq b)
	& \Longleftrightarrow (\neg\exists x_1,x_2\geq 0:A(x_1+x_2)\leq b)\\
	& \Longleftrightarrow (\neg\exists x'\geq 0:A'x'\leq b)\\
	& \Longleftrightarrow (\exists y\geq 0:y^TA'\geq 0,y^Tb<0)\\
	& \Longleftrightarrow (\exists y\geq 0:y^TA\geq 0,-y^TA\geq 0,y^Tb<0)\\
	& \Longleftrightarrow (\exists y\geq 0:y^TA= 0,y^Tb<0)
	\end{align*}
\end{problem}

\subsection{A Linear Program for, well, for what?}

Let $G = (V,E)$ be a directed graph, $s,t \in V$,  and $c: E \rightarrow \mathbf{R}^+$ be a cost 
function. We want to find an $s-t$-flow $f$ of value $1$. Every edge $e$ generates cost $f(e) \cdot c(e)$, and we want to minimize the overall cost. There are no capacity constraints.
We can easily write this as a linear program MCF (Minimum Cost Flow):
\begin{align*}
  \textrm{MCF}(G,s,t,c): \qquad
  \begin{array}{ll}
    \textnormal{minimize} \quad & \multicolumn{1}{l}{\sum_{e \in E} c(e) f(e)} \\
    \\
    \textnormal{subject to} \quad & \sum_{v \in V} f(v,t)  = 1 \\
					        & \sum_{u \in V} f(u,v) - \sum_{w \in V} f(v,w)  = 0  \quad \forall\ v \in V  \setminus \{s,t\}\\
					        \\
     & f(e)  \geq 0 \ \forall \ e \in E 
  \end{array}
\end{align*}
Note that we have $m$ variables, one variable $f(e)$ for each edge $e$.
The first constraint says that the value of the flow should be 1. The other constraints say that 
the inflow at $v$ should equal the outflow.

\begin{problem}{2}
	\statement
   Let $d$ be the shortest path distance from $s$ to $t$ in the directed graph $G$, where distance
   means sum of the $c(e)$ along the path. Show that $\opt(MCF) = d$.
   \textbf{Hint.} Make sure you show both $\leq$ and $\geq$.
    \solution
	To prove $d\geq \opt(MCF)$, first assume a shortest path $P$ in $G$.We have: 
	$$\sum_{e \in P} c(e)=d$$
	Then we define $f$ :
	\begin{equation}
	f(e)=
	\begin{cases}
	1& e\in P\\
	0& e\notin P
	\end{cases}
	\end{equation}
	So $f$ is a solution of $MCF$ and $d=val(f)\geq \opt(MCF)$.
	
	To prove $d\leq \opt(MCF)$,first assume a flow $f_0$ with $val(f_0)=\opt(MCF)$.
	Every time pick an edge$(u_0,v_0)$ with $f_0(u_0,v_0)>0$ but is minimal. 
	Because $f_0$ is a flow, there exist a path $P_1$ from $s$ to $t$ including $(u_0,v_0)$, which can be found with searching.
	Because $\forall ef_0(e)\geq f_0(u_0,v_0)$, all edges on $P_1$ has at least $f_0(u_0,v_0)$.
	Then generate a new flow $f_1$:
	\begin{equation}
	f_1(e)=
	\begin{cases}
	f_0(e)-f_0(u_0,v_0)& e\in P_1\\
	f_0(e)& e\notin P_1
	\end{cases}
	\end{equation}
	Because $P$ is a path and $f_0(e)\geq f(u_0,v_0)$, $f_1$ is still a flow but with $\sum_{v \in V} f_1(v,t) = 1-f_0(u_0,v_0)$.
	Notice that $f_1(u,v)=0$, then the edges included in $f_1$ is decreased by at least 1, so this procedure can terminate in finite times.
	By this way $f_0$ is divided into many path $P_i$ with their weight 
	$w_i=f_{i-1}(u_{i-1},v_{i-1})$ , and $\sum w_i=1$. 
	Because for every path $d\leq dis(P)=w*\sum_{e \in P} c(e)$,there is:
	\begin{align*}
	\opt(MCF)&=val(f_0)\\
			&=val(f_1)+w_1*\sum_{e \in P_1} c(e)\\
			&=\cdots\\
			&=\sum (w_i*\sum_{e \in P_i} c(e))\\
			&\geq \sum(w_i*d)\\
			&=d
	\end{align*}
	So $d\leq \opt(MCF)$.
\end{problem}

\begin{problem}{3}
	\statement
    Write down the dual of MCF. This will be a maximization problem. Don't use any matrix notation.
    \solution
    The original LP form is:
	\begin{align*}
	\text{minimize } & \sum_{e\in E}c(e)f(e) \\
	\text{subject to } & \sum_{e\in E,e=(u,t)}f(e)\geq 1 \\
	& \sum_{e\in E,e=(u,t)}-f(e)\leq -1 \\
	& \sum_{e\in E,e=(u,v)}f(e)-\sum_{e\in E,e=(v,w)}f(e)\geq 0, 
	\forall v\in V\setminus \{s,t\} \\
	& \sum_{e\in E,e=(v,w)}f(e)-\sum_{e\in E,e=(u,v)}f(e)\geq 0, 
	\forall v\in V\setminus \{s,t\} 
	\end{align*}
	Then the dual is:
	\begin{align*}
	\text{maximize } & y_t^+-y_t^- \\
	\text{subject to } & (y_v^+-y_v^-)-(y_u^+-y_u^-)\leq c(e),\forall e=(u,v),u\neq s\\
	& (y_v^+-y_v^-)\leq c(e),\forall e=(s,v)\\
	& y_v^+,y_v^-\geq 0,\forall v\in V\setminus s 
	\end{align*}
	It can be simplified to:
	\begin{align*}
	\text{maximize } & z_t \\
	\text{subject to } & z_v-z_u\leq c(e),\forall e=(u,v),u\neq s\\
	& z_v\leq c(e),\forall e=(s,v)\\
	& z_v\in R,\forall v\in V\setminus s 
	\end{align*}
	Then define the new $z_v=z_v+z_s$, it turns to:
	\begin{align*}
	\text{maximize } & z_t-z_s \\
	\text{subject to } & z_v-z_u\leq c(e),\forall e=(u,v)\\
	& z_v\in R,\forall v\in V 
	\end{align*}
\end{problem}
\begin{problem}{4}
	\statement
   Interpret the dual. Show that it is the LP formulation of a ``natural'' maximization problem on $G$.
   \solution
	For every vertex $v\in V$, give a real value $z_v$ so that for every $e=(u,v)$ holds $z_v-z_u\leq c(e)$. The target is to maxmize the difference of $z_s$ and $z_t$.
\end{problem}
\begin{problem}{5}
	\statement
  Describe an optimal solution of the dual program.
    \solution
	An optimal solution is:
	\begin{equation}
	f_v=
	\begin{cases}
	0& v=s\\
	\text{length of the shortest path from s to v}&v\neq s
	\end{cases}
	\end{equation}
	It satisfies the conditions of $f_v-f_u\leq c(e),\forall e=(u,v)$ and $f_v\in R,\forall v\in V$, so $f$ is a solution.
	Assume a shortest path $P=(e_1,e_2,\cdots,e_n)$ from $s$ to $t$,with vertices$(s,v1,v2,\cdots,t)$ then:
	\begin{align*}
	z_t-z_s&=(z_t-z_vn-1)+(z_vn-1-z_vn-2)+\cdots+(z_v1-z_s)\\
	&\leq c(e_1)+c(e_2)+\cdots+c(e_n)\\
	&=\text{length of the shortest path from s to t}\\
	&= f(t)-f(s)
	\end{align*}
	So $f$ is optimal.
\end{problem}


\end{document}